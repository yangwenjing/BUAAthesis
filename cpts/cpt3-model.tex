\chapter{海量数据分析与车辆移动模型建立}

本文的主要研究目标有:

   \textbf{发现海量交通数据中的出租车移动特征。}包括车辆移动特性与时间、空间以及自身状态的关系。本文从实际数据出发,观察车辆轨迹随时间,空间的变化情况。分析了车辆速度,持续时长等指标随时间,空间,车辆状态变化是否存在规律性。基于这些分析,建立移动模型。
   
   \textbf{建立移动模型.}宏观方面,建立区域转移矩阵,规定车辆从当前区域到其他区域的移动概率分布。最简单的区域定义是将场景划分成规则的四边形,我们认为这样不能体现出不同区域在出租车在地理上的不同点。
因此,我们将区域划分为细粒度的网格,对相邻的具有相似上客(下客)量的网格聚类为区域。
微观方面,抽象实际地图,实现基于地图的最短路径寻径算法。从实际数据中分时间段获得速度分布,拟合获得对应时段的速度分布函数,产生节点运动速度。因为速度字段是GPS采样数据,不准确,为了减小误差,我们采用一段重车(空车)距离除以时间的方法获取速度,然后研究其速度的分布情况。
基于ONE (Opportunity network environment )仿真平台,实现移动模型。

  \textbf{验证模型的有效性.}
通过文献阅读,我们选取了两个角度的指标,分别是出入度和接触指标。以此来验证模型是否具有较强的真实性。


根据以上研究目标,本文的研究内容也可以分为以下三部分。
\begin{itemize}
  \item 发现海量交通数据中的出租车移动特征:
  \begin{itemize}
  \item 宏观统计出租车整体的速度、各个状态的速度、状态持续时间等。
  \item 统计各个时段出租车的分布情况,以及载客、下客事件的分布情况。
  \end{itemize}
 \item 基于统计规律,建立移动模型:
 \begin{itemize}
 \item 探究车辆载客和空载区域的不同,对区域进行聚类,分别划分载客区域集和下客区域集。
 \item 计算载客区域到下客区域的转移概率以及下客区域到载客区域的转移概率。
 \item 对出租车载客状态和空车状态的速度进行建模。
 \item 确定出租车的起点和目的地,确定寻径策略,拟采取的寻径策略是最短路径算法。
 \end{itemize}
 \item 验证模型的有效性:
  \begin{itemize}
  \item 选取多角度指标,对模型的有效性进行验证,拟从接触角度和节点分布角度对模型进行验证。
  \item 与真实轨迹比较,确定出租车移动模型与实际的相似性。
  \item 与其他移动模型比较,确定基于海量数据挖掘的移动模型在准确性在优于其他移动模型。
 \end{itemize}
\end{itemize}

\section{海量数据分析}
\section{建立模型}
移动模型定义了节点的运动模式$Paths:<p_1,p_2…,p_n>$,$p_i$的确定可以简化为两步,即,目的地点选择和从源地点到目的地点的移动模式。
目的地点选择:节点的目的选择也与节点的当前状态有关。若节点处于载客状态。若当前状态为载客状态:针对载客事件将区域划分为不同的子区域,同理由下客事件分布将区域划分为不同的子区域。计算由载客子区域到下客子区域的转移概率矩阵和距离范围。然后由节点当前位置,决定下客位置。同理,节点处于下客状态时,由当前位置,和区域转移矩阵也可以计算得到载客的目的节点。
\subsection{区域识别}
lalal


\subsection{区域转移概率}
区域转移概率


\subsection{时间建模}
time~


\subsection{速度建模}
\begin{figure}[!h]
\centering
\begin{tabular}
[c]{cc}
\multicolumn{2}{c}{6:00-8:00}\\
\epsfysize=1.5in\epsfbox{figures/evalue/fitspeed6_0.eps} &
\epsfysize=1.5in\epsfbox{figures/evalue/fitspeed6_1.eps} \\
\multicolumn{2}{c}{11:00-13:00}\\
\epsfysize=1.5in\epsfbox{figures/evalue/fitspeed11_0.eps} &
\epsfysize=1.5in\epsfbox{figures/evalue/fitspeed11_1.eps} \\
\multicolumn{2}{c}{17:00-19:00}\\
\epsfysize=1.5in\epsfbox{figures/evalue/fitspeed17_0.eps} &
\epsfysize=1.5in\epsfbox{figures/evalue/fitspeed17_1.eps} \\
\multicolumn{2}{c}{22:00-24:00}\\
\epsfysize=1.5in\epsfbox{figures/evalue/fitspeed22_0.eps} &
\epsfysize=1.5in\epsfbox{figures/evalue/fitspeed22_1.eps} \\
(a) vacant status & (b) occupied status \\
\end{tabular}
\caption{速度分布的拟合结果}\label{figure_fitspeed_varied_with_time}
\end{figure}



为了获取各个状态的速度分布,我们对瞬时速度的累积分布进行拟合,以获取瞬时速度的累积分布函数,然后衍生出其速度分布函数。
由图\ref{figure_fitspeed_varied_with_time}可知,除了载客状态时从22:00-24:00的累积瞬时速度分布外,瞬时速度的累积分布表现出指数分布的规律,拟合函数记为$f_1(x)$。载客状态时从22:00-24:00的累积瞬时速度分布表现出线性分布的特性,其拟合函数记为$f_2(x)$, 如公式\ref{formular_ccdf_speed}。由分析过程可知,某些天,例如周末的某些时段会影响车辆的行为,我们仅分析最常见的情况,因此去掉了明显不一样的情况,例如周六和周天早上6点到8点时的载客状态的累积速度分布。

\begin{equation}\label{formular_ccdf_speed}
\left\{
\begin{array}{ll}
f_1(x) = 1-1/exp(-ax^b-c)\\
f_2(x) = ax+b
\end{array}
\right.
\end{equation}

\begin{table}[ht]
\caption{拟合参数以及拟合曲线的残差平方和}\label{table_rms}
\centering
\begin{tabular}{c|c|c}
  \hline
  时间段 & 空车状态 & 载客状态 \\
  \hline
6:00-8:00   &0.0129207 & 0.019818 \\
11:00-13:00 &0.00866176 & 0.0204889 \\
17:00-19:00 &0.0176578 & 0.0105868 \\
22:00-24:00 &0.0154822 & 0.0240426 \\
  \hline
\end{tabular}
\end{table}

拟合结果以及相关的残差平方和(root mean square, rms)如表\ref{table_rms}所示,越小的残差平方和代表越小的误差。由表\ref{table_rms}可知,所有的残差平方和均小于$0.025$,表现出较好的拟合相似性。


\section{小结}
tete

