区域转移概率是指从一个区域到另一个区域的概率,本文中的区域转移概率依据事件分别划分成两组区域集合。例如,如果一辆车在某地点发生了载客事件,它的下一个事件为下客事件,那么区域转移概率就是从当前地点和事件(载客)找到对应的(载客)区域,然后目的地点和(下客)事件将在相对的(下客)区域集合中选择。
由于载客和下客事件的分布随时间变化而变化,且与事件类型有关,我们分别对时间$t$,载客区域集$R^{load}_{i,t}$和下客区域集$R^{drop}_{j,t}$按照对应的事件分布分别被识别出来。从时间$t$开始,由载客区域$i$到下客区域$j$的转移概率可表示为:

\begin{equation}
p^{load\rightarrow drop}_{i\rightarrow j,t}
\end{equation}

相似的,从时间$t$开始,由下客区域$j$到载客区域$i$的转移概率可表示为:

\begin{equation}
p^{drop\rightarrow load}_{j\rightarrow i,t}
\end{equation}

通过\textbf{区域识别}过程,一段时间内的每个区域内的车辆集合都可以被标记出来。由于从载客区域到下客区域的转移概率的计算和从下客区域到载客区域的的转移概率的计算方法类似,我们这里仅详细解释从载客区域到下客区域的转移概率的计算。

为了计算$p^{drop\rightarrow load}_{j\rightarrow i,t}$, 我们需要找到从$t$时间段在区域$j$中的所有的出租车集合,记为$V^{drop}_{j,t}$. 对于每一辆出租车$v \in V^{drop}_{j,t}$, 其下一次载客事件发生的记录(包含载客时间,地点等信息)可以被找到。我们将所有车辆的下一跳记录按照时间和地点进行映射,找出多少在区域$i$中。由于有些纪录的缺失,下一跳的纪录和本次纪录的间隔时间较长,因此,我们限定同一辆车的两次相邻纪录的时间间隔不能跨越两个时间区间。例如,设时间区间为1小时,那么第一条纪录的时间为13点半,第二条纪律的时间为14点,那么这两条纪录仅跨了一个时间区间。而如果第二条纪录的时间为15点,那么就认为这条纪录跨过了两个时间区间。

对应的,区域转移概率矩阵也是与时间相关的,对于1小时的时间间隔,需要设计的载客区域和空载区域都需要计算两个时间间隔内的转移概率矩阵。并进行归一化。

