% !Mode:: "TeX:UTF-8"

% 中英文摘要
\begin{cabstract}
在交通领域中,交通瓶颈的预测以及寻径算法的验证大都建立在仿真环境中。类似的,车联网中路由协议的验证也是建立在仿真平台上。而其中的难点在于如果提高仿真的准确性。移动模型是现有的车辆仿真的基础,对提高仿真的准确性具有重要意义。基于海量交通数据挖掘的出租移动模型是基于北京市12000多辆出租车的移动轨迹进行分析建模。首先由日常观察和经验出发,提出假设,假设出租车的移动行为与出租车载客和空载状态,时间以及车辆地理位置有关。然后分别从车辆状态,时间和地理位置角度分析北京市的出租车数据,分析结果证明我们提出的三个假设合理。然后对北京市出租车从宏观和微观角度建模。宏观角度,我们按照时间和载客/下客事件划分区域并基于海量交通数据分别计算载客区域到下客区域以及下客区域到载客区域的转移概率矩阵。微观方面,抽取北京市地图,定义寻径算法以及对车辆速度建模。由于我们考虑到了车辆的状态和时间因素,区域划分和区域转移概率矩阵会随时间和状态的变化而相应变化。在实验验证方面,我们验证了车辆的接触指标以及车辆的出入度指标,并与实际数据,随机路点移动模型最短路径移动模型相比较。实验证明相对于其他移动模型,基于海量交通数据的移动模型和实际数据的接触特征最为相似。出入度方面,每天的实际轨迹出入度与平均实际轨迹区域出入度的相对误差仅为$10\%$,本文提出的移动模型与实际轨迹平均区域出入度的相对误差为$47\%-49\%$,而其他移动模型的相对误差达到$65\%-69\%$和$81\%-83\%$,准确性提高了$10\%-40\%$.实验结果表明本文提出的移动模型具有较高的仿真相似性。

\end{cabstract}

\begin{eabstract}
The mobility model is one of the most important factors that impacts the evaluation of any transportation vehicular networking protocols via simulations. However, to obtain a realistic mobility model in the dynamic urban environment is very challenging. Recently, several studies extract mobility models from large-scale real data sets (mostly taxi GPS data) without consideration of the statuses of taxi. In this paper, we discover three simple observations related to the taxi statuses via mining of real taxi traces: (1) the behavior of taxi will be influenced by the statuses, (2) the macroscopic movement is related with different geographic features in corresponding status, and (3) the taxi load/drop events are varied with time. 
Based on these three observations, a novel taxi mobility model (T-START) is proposed with respect to taxi statuses, geographic region and time. The simulation results illustrate that proposed mobility model has a good approximation with reality in trace samples and distribution of nodes in four typical time period.
\end{eabstract}