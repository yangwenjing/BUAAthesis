% !Mode:: "TeX:UTF-8"

% 中英文摘要
\begin{cabstract}
本篇文档主要介绍{\bf 北航毕业设计论文\LaTeX{}模板}使用和相关软件环境的安装配置,
以及本模板所遵循的开源协议等。
\end{cabstract}

\begin{eabstract}
The mobility model in urban vehicular networks is one of the most important factors that impacts the evaluation of any vehicular networking protocols via simulations. However, to obtain a realistic mobility model in the dynamic urban environment is very challenging. Recently, several studies extract mobility models from large-scale real data sets (mostly taxi GPS data) without consideration of the statuses of taxi. In this paper, we discover three simple observations related to the taxi status via mining of real taxi traces: (1) the behavior of taxi will be influenced by the statuses, (2) the macroscopic movement is related with different geographic features in corresponding status, and (3) the taxi load/drop events are varied with time. 
Based on these three observations, a novel taxi mobility model (T-START) is proposed with respect to taxi statuses, geographic region and time. The simulation results illustrate that proposed mobility model has a good approximation with reality in the contact characteristics, trace samples and distribution of nodes in four typical time period.
\end{eabstract}